\documentclass[a4paper,10pt,article]{memoir}

\usepackage[a4paper,margin=1in]{geometry}

\usepackage[utf8]{inputenc}
\usepackage{fourier}
\usepackage{amsmath,amssymb}

\usepackage{color}
\definecolor{orange}{rgb}{0.75,0.5,0}
\definecolor{magenta}{rgb}{1,0,1}
\definecolor{cyan}{rgb}{0,1,1}
\definecolor{grey}{rgb}{0.25,0.25,0.25}
\newcommand{\outline}[1]{{\color{grey}{\scriptsize #1}}}
\newcommand{\revnote}[1]{{\color{red}\textit{\textbf{#1}}}}
\newcommand{\note}[1]{{\color{blue}\textit{\textbf{#1}}}}
\newcommand{\citenote}[1]{{\color{orange}{[\textit{\textbf{#1}}]}}}

\usepackage{xspace}

\usepackage{listings}
\usepackage{courier}
\lstset{basicstyle=\tiny\ttfamily,breaklines=true,language=Python}

\usepackage{fancyvrb}
\usepackage{url}

\usepackage{float}
\newfloat{listing}{tbp}{lop}
\floatname{listing}{Listing}

\title{GENIE \texttt{carrotcake} on Windows}
\author{Ian~Ross}
\date{4 March 2015}

\begin{document}

\catcode`~=11    % Make tilde a normal character (danger of weirdness...)
%\catcode`~=13    % Make tilde an active character again

\maketitle

This document describes how to use the new \texttt{carrotcake} version of
the GENIE model on Windows.  Documentation is divided into two
sections, one for users of the model and one for those who wish to
modify the model.

\emph{Throughout this document, shell commands are shown in
  \texttt{typewriter} font.  Commands that extend over several lines
  are marked with lines ending \texttt{...} with the following line
  beginning \texttt{...}.}

%======================================================================
\chapter{For GENIE users}

%----------------------------------------------------------------------
\section{Installation and setup}

The following prerequisites are needed in order to install and use
GENIE on Windows:
\begin{description}
  \item[Git]{You can install Git from here:
    \url{http://git-scm.com/download/win}.}
  \item[Python]{You need Python version 2.7.9, which can be installed
    from here:
    \url{https://www.python.org/downloads/release/python-279/}.}
  \item[Microsoft Visual Studio]{All of the features described here
    have been tested with \emph{Microsoft Visual Studio Ultimate 2013}
    (version 12.0.31101.00).}
  \item[Intel Visual Fortran]{All of the features described here have
    been tested with \emph{Intel Parallel Studio XE 2015 Composer
      Edition for Fortan} (compiler version 15.0.2.179).}
\end{description}

I have no way of testing with other versions of Visual Studio or Intel
Fortran, but significantly older versions of either are unlikely to
work.

To install GENIE, first choose a location for the installation.  On
Windows, it's probably best just to install GENIE in
\texttt{C:\textbackslash{}cgenie}.  In a command prompt window, clone
the main \texttt{cgenie} repository from GitHub using the command
\begin{verbatim}
  git clone https://github.com/genie-model/cgenie.git
\end{verbatim}
This will produce a new directory called \texttt{cgenie} containing
the model source code and build scripts.

You then need to install NetCDF libraries suitable for use on Windows:
\begin{verbatim}
  c:
  cd \cgenie\tools
  git clone https://github.com/genie-model/netcdf-min-ifort-win netcdf
\end{verbatim}
and you need to add the path to the NetCDF Fortran 90 DLL
(\texttt{C:\textbackslash{}genie\textbackslash{}tools\textbackslash{}netcdf\textbackslash{}ia32\textbackslash{}bin})
to your \texttt{PATH}.

Before using the model, it's necessary to do a little bit of setup.
Go into the new \texttt{cgenie} directory and run the
\texttt{setup-cgenie} script:
\begin{verbatim}
  c:
  cd \cgenie
  setup-cgenie
\end{verbatim}
Note that unlike the Linux setup scripts, this does \emph{not} test
for a suitable version of Python -- it's up to you to make sure that
you have Python 2.7.9 installed!  The script will ask where you want
to put a number of things -- it's usually find to just take the
defaults (just hit enter at each of the prompts).  In all of what
follows below, we'll assume that you chose the defaults.  The things
the script asks for are:
\begin{itemize}
  \item{The GENIE root installation directory: unless you know what
    you're doing, accept the \texttt{C:\textbackslash{}cgenie} default
    for this.}
  \item{The GENIE data directory (default
    \texttt{C:\textbackslash{}cgenie-data}) where base and user model
    configurations are stored, along with forcing files.}
  \item{The GENIE test directory (default
    \texttt{C:\textbackslash{}cgenie-test}) where GENIE jobs with
    known good outputs can be stored for use as tests -- it's possible
    to run sets of tests and compare their results with the known good
    values with a single command, which is useful for making sure that
    the model is working.}
  \item{The GENIE jobs directory (default
    \texttt{C:\textbackslash{}cgenie-jobs}) where new GENIE jobs are
    set up -- the \texttt{new-job} script (see next section) sets jobs
    up here by default.}
  \item{The default model version to use for running jobs.  By
    default, the most recent released version is selected, but you can
    type another version if necessary.  You can also set jobs up to
    use any model version later on.}
\end{itemize}
After providing this information, the setup script will ask whether
you want to download the data and test repositories.  It's usually
best to say yes.

Once the data and test repositories have been downloaded, GENIE is
ready to use.  The setup information is written to a
\texttt{.cgenierc} file in your home directory (which can be in a
number of places on Windows: the \texttt{USERPROFILE} environment
variable normally points to the right place).  If you ever want to set
the model up afresh, just remove this file and run the
\texttt{setup-cgenie} script again.

\paragraph{Important:}

There are a large number of environment variables that need to be set
up for the Intel Fortran compiler to work correctly.  You \emph{must}
run the GENIE configuration and build scripts from a command prompt
window that has these environment variables set.  Command prompt
shortcuts with the appropriate variables set will be found in the
Start menu under: ``All Programs'' $\to$ ``Intel Parallel Studio XE
2015'' $\to$ ``Compiler and Performance Libraries'' $\to$ ``Command
Prompt with Intel Compiler XE v15.0 Update 2'' $\to$ ``IA-32 Visual
Studio 2013 mode''.  \emph{Note that only 32-bit builds of GENIE are
  currently supported.}

\vspace{\baselineskip}

To check that the installation has been successful and that the model
works on your machine, you can run some basic test jobs -- in the
\texttt{C:\textbackslash{}cgenie} directory, just type
\begin{verbatim}
  tests run basic
\end{verbatim}
This runs an ocean-atmosphere simulation and an ocean biogeochemistry
simulation using the default model version selected at setup time.

%----------------------------------------------------------------------
\section{Creating new jobs}

New GENIE jobs are configured using the \texttt{new-job} script in
\texttt{~/cgenie}.  This takes a number of arguments that describe the
job to be set up and produces a job directory under
\texttt{~/cgenie-jobs} containing everything needed to build and run
the model with the selected configuration.  The \texttt{new-job}
script should be run as:
\begin{verbatim}
  new-job [options] job-name run-length
\end{verbatim}
where \texttt{job-name} is the name to be used for the job directory
to be created under \texttt{C:\textbackslash{}cgenie-jobs} and
\texttt{run-length} is the length of the model run in years.  The
possible options for \texttt{new-job} are as follows (in each case
given in both short and long forms where these exist).  First, there
are three options that control the basic configuration of the model.
In most cases, a base and a user configuration should be supplied
(options \texttt{-b} and \texttt{-u}).  In some special circumstances,
a custom ``full'' configuration may also be used (the \texttt{-c}
option).

\begin{verbatim}
  -b BASE_CONFIG   --base-config=BASE_CONFIG
\end{verbatim}
The base model configuration to use -- these are stored in the
\texttt{C:\textbackslash{}cgenie-data\textbackslash{}base-configs} directory.

\begin{verbatim}
  -u USER_CONFIG   --user-config=USER_CONFIG
\end{verbatim}
The user model configuration to apply on top of the base configuration
-- model user configurations are stored in the
\texttt{C:\textbackslash{}cgenie-data\textbackslash{}user-configs}
directory.

\begin{verbatim}
  -c CONFIG        --config=CONFIG
\end{verbatim}
Full configuration name (this is mostly used for conversions of
pre-\texttt{carrotcake} tests) -- full configurations are stored in the
\texttt{C:\textbackslash{}cgenie-data\textbackslash{}full-configs}
directory.

In addition to the configuration file options, the following
additional options may be supplied to \texttt{new-job}:

\begin{verbatim}
  -O     --overwrite
\end{verbatim}
Normally, \texttt{new-job} will not overwrite any existing job of the
requested name.  Supplying the \texttt{-O} flag causes
\texttt{new-job} to delete and replace any existing job with the
requested name.

\begin{verbatim}
  -r RESTART    --restart=RESTART
\end{verbatim}
One GENIE job can be \emph{restarted} from the end of another.  This
option allows for a restart job to be specified.  This must be a job
that has already been run (so that there is output data to use for
restarting the model).

\begin{verbatim}
  --old-restart
\end{verbatim}
It may sometimes be useful to restart from an old pre-\texttt{carrotcake}
job.  This flag indicates that the job name supplied to the
\texttt{-r} flag is the name of an old GENIE job whose output can be
found in the \texttt{C:\textbackslash{}cgenie\_output} directory.

\begin{verbatim}
  --t100
\end{verbatim}
This flag indicates that the job should use the alternative "T100"
timestepping options for the model (i.e. 100 timesteps per year for
the default model resolution instead of 96).

\begin{verbatim}
  -j JOB_DIR     --job-dir=JOB_DIR
\end{verbatim}
It can sometimes be useful to put GENIE jobs somewhere other than
\texttt{C:\textbackslash{}cgenie-jobs}.  This flag allows an
alternative job directory to be specified.

\begin{verbatim}
  -v MODEL_VERSION    --model-version=MODEL_VERSION
\end{verbatim}
Normally, \texttt{new-job} will generate a job set up to use the
default model version which was selected when the
\texttt{setup-cgenie} script was run.  This flag allows for a
different model version to be selected.

\subsection*{Examples}

\begin{verbatim}
  new-job -b cgenie.eb_go_gs_ac_bg.p0650e.NONE ...
            ... -u LABS\LAB_0.snowball snowball 10
\end{verbatim}
This configures the first example job in the workshop handout.  After
running this invocation of \texttt{new-job}, a new
\texttt{~/cgenie-jobs/snowball} job directory will have been created
from which the job can be executed.

\begin{verbatim}
  new-job -b cgenie.eb_go_gs_ac_bg.p0650e.NONE ...
            ... -u LABS\LAB_0.snowball -r snowball snowball2 10
\end{verbatim}
This invocation of \texttt{new-job} sets up a new \texttt{snowball2}
job that restarts from the end of the \texttt{snowball} job to run for
an additional 10 years.

%----------------------------------------------------------------------
\section{Running jobs}

Once a job has been set up using the \texttt{new-job} script, it can
be run from the newly created job directory using a ``\texttt{go}''
script.  Configuring and running a job is as simple as:
\begin{verbatim}
  c:
  cd \cgenie
  new-job -b cgenie.eb_go_gs_ac_bg.p0650e.NONE ...
            ... -u LABS/LAB_0.snowball snowball 10
  cd \cgenie-jobs\snowball
  go run
\end{verbatim}
The \texttt{go} script has three main options and two advanced
options.  The basic options are:
\begin{description}
  \item[\texttt{go clean}]{Remove model output and model executables
    and compiled object files for the current job setup.}
  \item[\texttt{go build}]{Compile the required version of the model
    to run this job -- this depends on a number of things, including
    the selected model resolution, but the build system ensures that
    model executables are not recompiled unnecessarily.}
  \item[\texttt{go run}]{Compile the model (if necessary) and run
    the current job.}
\end{description}
Both the \texttt{build} and \texttt{run} commands can also take a
``build type'' argument for building debug or profiling versions of
the model.  For more information about this and about how the build
system maintains fresh executables of selected versions of the model,
see Section~\ref{sec:genie-devs}.  Also see that section for the two
``advanced'' options to the \texttt{go} script, which are used to
select alternative ``platforms'' for a machine -- in the normal case,
the build system will select the appropriate compilers and flags based
on the machine on which the model is being run (assuming that a
platform definition has been set up for the machine), but sometimes it
may be desirable to select between different compilers on the same
machine, for which a \texttt{set-platform} option is provided by the
\texttt{go} script.

%----------------------------------------------------------------------
\section{Managing configuration files}

Configuration files are all kept in
\texttt{C:\textbackslash{}cgenie-data}, base configurations in the
\texttt{base-configs} directory and user configurations in
\texttt{user-configs}.  All of this configuration data is held in a
Git repository on GitHub, so if you want to add user or base
configurations to share with other users, ask someone about how to set
yourself up to use GitHub.

%----------------------------------------------------------------------
\section{Managing tests}

It is possible to save job configurations and results as test jobs
with ``known good'' data.  This has two main uses -- first, for
testing a GENIE installation to make sure that it's working; second,
to test that changes to the model don't inadvertently affect
simulation results.  The second application is of more interest for
people changing the GENIE model code, but it can still be useful to
save jobs as tests.

The \texttt{tests} script in \texttt{C:\textbackslash{}cgenie} is used
to manage and run test jobs.  To list the available tests, do
\begin{verbatim}
  tests list
\end{verbatim}
and to run an individual test or a set of tests, do
\begin{verbatim}
  tests run <test>
\end{verbatim}
where \texttt{<test>} is either a single test name (e.g.
\texttt{basic/biogem}), a set of tests (e.g. \texttt{basic}) or
\texttt{ALL}, which runs \emph{all} available tests.  The tests are
run as normal GENIE jobs in a subdirectory of
\texttt{C:\textbackslash{}cgenie-jobs} with a name of the form
\texttt{test-YYYYMMDD-HHMMSS} based on the current date and time.  As
well as full test job output, build and run logs, a \texttt{test.log}
file is produced in this test directory, plus a \texttt{summary.txt}
file giving a simple pass/fail indication for each test.

An existing job can be added as a test using a command like
\begin{verbatim}
  tests add <job-name>
\end{verbatim}
where \texttt{<job-name>} is the name of an existing job in
\texttt{C:\textbackslash{}cgenie-jobs}.  Note that you need to run the
job before you can add it as a test!  The test script will ask you
which output files you want to use for comparison for each model
component -- there are sensible defaults in most cases, but you can
select individual files too if you prefer.

There are two other features of the test addition command that can be
useful.  First, it's possible to give the test a different name than
the job it's made from -- for example
\begin{verbatim}
  tests add hosing\test-1=hosing-experiment-1
\end{verbatim}
adds a test called \texttt{hosing\textbackslash{}test-1} based on the
\texttt{hosing-experiment-1} job.  Second, it's possible to say that a
new test should be restarted from the output of an existing test.
Normally, if a test is created from a job that requires restart files,
the restart files are just copied from the job into the new test.
Sometimes though, it can be of interest to run the job that generated
the restart data, then immediately run a test starting from the output
of the first test.  This can be done using something like this:
\begin{verbatim}
  tests add foo\test-1=job-1
  tests add foo\test-2=job-2 -r foo\test-1
\end{verbatim}
This indicates that \texttt{foo\textbackslash{}test-1} is a ``normal''
test, while \texttt{foo\textbackslash{}test-2} is a test that depends
on \texttt{foo\textbackslash{}test-1} for its restart data.  When you
run a test that depends on another for restart data, the test script
deals with making sure that the restart test is run before the test
that depends on it.  So, for example, you can just say
\begin{verbatim}
  tests run foo\test-2
\end{verbatim}
and the test script will figure out that it needs to run
\texttt{foo\textbackslash{}test-1} first in order to generate restart
data for \texttt{foo\textbackslash{}test-2}.

%----------------------------------------------------------------------
\section{Managing model versions}

For most users, it makes sense to run jobs using the most recent
available version of the GENIE model code.  This is the option chosen
by default when the model is initially set up.  However, it can
sometimes be useful to run jobs with earlier model versions (or with a
development version of the model -- see the next section).  The GENIE
configuration and build system provides a simple mechanism to permit
this, hiding most of the (rather complex) details of managing multiple
model versions from users.

Model versions are indicated by Git ``tags''.  In order to see a list
of available model versions, use the following command in the
\texttt{C:\textbackslash{}cgenie} directory:
\begin{verbatim}
  git tag -l
\end{verbatim}
To configure a job to use a different model version from the default,
simply add a \texttt{-v} flag to \texttt{new-job} specifying the model
version you want to use.  For example, to configure a job to use the
\texttt{carrotcake-1.0} version of the model, use something like the
following command:
\begin{verbatim}
  new-job -b cgenie.eb_go_gs_ac_bg.p0650e.NONE ...
            ... -u LABS\LAB_0.snowball snowball 10 ...
            ... -v carrotcake-1.0
\end{verbatim}
Within a job directory, you can see what model version the job was
configured with by looking at the contents of the
\texttt{config\textbackslash{}model-version} file -- in
non-development cases, this will just contain the Git tag of the model
version.


%======================================================================
\chapter{For GENIE developers}
\label{sec:genie-devs}

For developers of GENIE, there are a few extra things to know beyond
what's needed to run the model.  Most of this is covered in the
\emph{GENIE \texttt{carrotcake} Configuration and Build System} document.

The only Windows-specific feature to be covered here is how to debug
the model in Visual Studio.  This is a little complicated and there
are a number of possible approaches.  The following steps describe the
recommended method:

\begin{enumerate}
  \item{Set up a new GENIE job for the conditions you want to use for
    debugging using the \texttt{new-job} script.}
  \item{Start Microsoft Visual Studio.}
  \item{Open the
    \texttt{C:\textbackslash{}cgenie\textbackslash{}cgenie-msvs\textbackslash{}cgenie-msvs.sln}
    Visual Studio solution file.}
  \item{Right click on the ``\texttt{cgenie-msvs}'' project line in
    the Solution Explorer and choose ``Properties''.}
  \item{In the properties dialog, select ``Configuration Properties''
    $\to$ ``Fortran'' $\to$ ``Preprocessor'' and edit the
    ``Preprocessor definitions'' line to include all the coordinate
    definitions listed in the \texttt{config\textbackslash{}job.py}
    file in the job directory of the job you want to use for
    debugging.}
  \item{Still in the properties dialog, select ``Configuration
    Properties'' $\to$ ``Debugging'' and edit the ``Working
    Directory'' field to point to the job directory of the job you
    want to use for debugging.}
  \item{Click ``OK'' in the properties dialog to save the changed
    settings.}
  \item{From the ``BUILD'' menu, select ``Rebuild Solution'' and wait
    for the model to be rebuilt.}
  \item{Set breakpoints in source files before starting debugging.}
  \item{From the ``DEBUG'' menu, select ``Start Debugging'' and debug
    as normal.  Output from GENIE will appear in a seperate console
    window.}
\end{enumerate}

\end{document}
